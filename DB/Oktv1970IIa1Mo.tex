Alakítsunk egy kicsit az eredetin:
\begin{align}
x^{3}-1-(a+b)(x-1) &= 2ab+a+b \label{asd}
(x-1)(x^{2}+x+1-(a+b)) &= 2ab+a+b
\end{align}
Ha $-1<x-1<0$ akkor $(x^{2}+x+1-(a+b))<-(2ab+a+b)$, 
ám ekkor $x^{2}+x+1<-2ab<0$, ami nem lehet. Ha $x-1>1$ akkor 
$0<(x^{2}+x+1-(a+b))< 2ab+a+b$-nak kell lenni, amiből $7<6$ következne.
$x=1$ esetén $a=b=0$ lenne, $x=2$-nél megint $7<6$-ot kapnánk.
