%Krs19612Mo
%A következő általános észrevételeket használjuk: 
%\par Legyen $x,y,z$  pozitív valós. Ha $xy>z>0$, akkor valamelyik nagyobb mint 
%$\sqrt{z}$.. Ha $xy>z$ és $x<\sqrt{z}$, akkor $y>\sqrt{z}$.
%\par Tegyük fel hogy mindannyian nagyobbak mint $\frac{1}{4}.$
%Ha $(1-a)b>\frac{1}{4}$, akkor $1-a>\frac{1}{2}$ vagy $b>\frac{1}{2}$. 
%Az első esetben a következő láncunk van:
%$$
%1-a>\frac{1}{2} \implies a<\frac{1}{2} \implies 1-c>\frac{1}{2} \implies 
%c < \frac{1}{2} \implies b < \frac{1}{2}.
%$$
%A másodikban:
%$$
%b>\frac{1}{2} \implies c > \frac{1}{2} \implies a > \frac{1}{2}.
%$$
%Tehát a feltételekből azt kapjuk hogy vagy mindannyian kisebbek, vagy mindannyian 
%nagyobbak $\frac{1}{2}$-nél. Ha mindannyian kisebbek, akkor:
%$$
%a=\frac{1}{2}-\delta_{a},\ b=\frac{1}{2}-\delta_{b},\ c=\frac{1}{2}-\delta_{c},
%$$
%pozitív $\delta_{*}$  valósokkal.

A számok felírhatók
$$
a=\frac{1}{2}+\delta_{a},\ b=\frac{1}{2}+\delta_{b},\ c=\frac{1}{2}+\delta_{c},
$$
alakban, alkalmas $-\frac{1}{2}<\delta_{a},\delta_{b},\delta_{c}<\frac{1}{2}$ 
valós számokkal. Tegyük fel hogy mindannyian nagyobbak mint $\frac{1}{4}.$
Ekkor:
\begin{align*}
(1-a)b=\Gzjel{\frac{1}{2}-\delta_{a}}\Gzjel{\frac{1}{2}+\delta_{b}}=\\ 
= \frac{1}{4}+\frac{\delta_{b}-\delta_{a}}{2}-\delta_{a}\delta_{b}>\frac{1}{4} 
\Leftrightarrow \delta_{b} > \delta_{a} + 2\delta_{a}\delta_{b}
\end{align*}
hasonlóan az $(1-b)c$ és $(1-c)a$ kifejezésekre:
\begin{equation}
\label{orig}
\tag{orig}
\begin{split}
\delta_{b} &> \delta_{a} + 2\delta_{a}\delta_{b} \\
\delta_{c} &> \delta_{b} + 2\delta_{b}\delta_{c} \\
\delta_{a} &> \delta_{c} + 2\delta_{a}\delta_{c}.
\end{split}
\end{equation}
%Ha valamelyik szám, pl. $a=\frac{1}{2}$, azaz $\delta_{a}=0$, akkor:
%$$
%\delta_{b}>0,\ 0>\delta_{c}=\delta_{b}(1+2\delta_{c}) \implies 
%\delta<-\frac{1}{2},
%$$
%ami lehetetlen. Feltehetjuk tehát, hogy 
Ezeket átrendezve, felhasználva $1-2\delta_{x}>0, \ x=a,b,c$-re, kapjuk hogy:
\begin{align*}
\delta_{b} &> \frac{\delta_{a}}{1 - 2\delta_{a}} \\
\delta_{c} &> \frac{\delta_{b}}{1 - 2\delta_{b}} \\
\delta_{a} &> \frac{\delta_{c}}{1 - 2\delta_{c}}.
\end{align*}
Innen könnyen látható, hogy a $\delta_{x}$-ek egyforma előjelűek. Ekkor (\ref{orig})-ot 
szemrevételezve: $\delta_{b}>\delta_{a}>\delta_{c}>\delta_{b}$ ellentmondásra jutunk.
