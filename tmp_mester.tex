%\documentclass[10pt,twoside,titlepage,reqno]{amsbook}
\documentclass[12pt]{amsbook}



\setcounter{tocdepth}{3}



\makeatletter
\def\l@subsubsection{\@tocline{2}{5pt}{6pc}{8pc}{}}
\makeatother

\makeatletter
\def\l@subsection{\@tocline{2}{2pt}{2.5pc}{5pc}{}}
\makeatother



\usepackage[margin=2.5cm,nohead]{geometry}
\usepackage[utf8]{inputenc}
\usepackage[hungarian]{babel}
\usepackage[%hypertex,
                 unicode=true,
                 plainpages = false, 
                 pdfpagelabels, 
                 bookmarks=true,
                 bookmarksnumbered=true,
                 bookmarksopen=true,
                 breaklinks=true,
                 backref=false,
                 colorlinks=true,
                 linkcolor = blue,
                 urlcolor  = blue,
                 citecolor = red,
                 anchorcolor = green,
                 hyperindex = true,
                 hyperfigures
]{hyperref}
\hypersetup{
 pdftitle={Feladatok},
 pdfauthor={Czylabson Asa},
 pdfsubject={Hold Föld Nap}
}
\usepackage{amsmath}
\usepackage{amsfonts}
\usepackage{amssymb}
\usepackage{graphicx}
\usepackage{type1cm}

\usepackage{setspace}



\newcommand{\Fa}[1]{%
\begin{subsubsection}{#1}%
   \label{#1Fa}
   \hspace{0.5cm}
   \par
   \input{DB/#1Fa.tex}
   \hspace{0.5cm}
   \newline
   \par{(Um:\ref{#1Um})(Mo:\ref{#1Mo})}
   \newline
   \centerline{$\triangle\bigtriangledown\triangle$}
   \newline
\end{subsubsection}%
}%

\newcommand{\Um}[1]{%
\begin{subsubsection}{#1}%
   \label{#1Um}
   \hspace{0.5cm}
   \par
   \input{DB/#1Um.tex}
   \hspace{0.5cm}
   \newline
   \par{(Fa:\ref{#1Fa})(Mo:\ref{#1Mo})}
   \newline
   \centerline{$\triangle\bigtriangledown\triangle$}
   \newline
\end{subsubsection}%
}%

\newcommand{\Mo}[1]{%
\begin{subsubsection}{#1}%
   \label{#1Mo}
   \hspace{0.5cm}
   \par
   \input{DB/#1Mo.tex}
   \hspace{0.5cm}
   \newline
   \par{(Fa:\ref{#1Fa})(Um:\ref{#1Um})}
   \newline
   \centerline{$\triangle\bigtriangledown\triangle$}
   \newline
\end{subsubsection}%
}%


%---------------------------------------------------------------------
\newcommand{\Gzjel}[1]{%
{ \left( #1 \right) }
}
\newcommand{\Toligv}[2]{%
#1,\hdots ,#2
}

\newcommand{\Tolig}[2]{%
#1\hdots #2
}



%\makeindex
\begin{document}
\begin{spacing}{1.3}



% cím
\author{Czylabson Asa}
\title{Egyenlőtlenséges feladatok}
\maketitle


\pagestyle{plain}
%nem nagyon akarja a cimhez rakini a datumaot
\begin{center}{\bf\today}\end{center}
\tableofcontents
\clearpage
%\mainmatter
%\frontmatter
\newpage

%ez kell a jelölésekhez
%\item Schev {\ \ \ Schultz János: Elemi matematikai versenyfeladatok, könyv, IV fejezet, Egyenlőtlenségek}



%%%%%%%%%%%%%%%%%%%%%%%%%%%%%%%%%%%%%%%%%%%%%%%%%%%%%%%%%%%%%%%%%%%%
\hspace{2cm}
%%%%%%%%%%%%%%%%%%%%%%%%%%%%%%%%%%%%%%%%%%%%%%%%%%%%%%%%%%%%%%%%%%%%

\begin{section}{Feladatok}
   \label{Fa}
   \Fa{Schev1}
	\Fa{FolProd1}
   \hspace{0.5cm}
\newpage
\end{section}

%%%%%%%%%%%%%%%%%%%%%%%%%%%%%%%%%%%%%%%%%%%%%%%%%%%%%%%%%%%%%%%%%%%%
\hspace{2cm}
%%%%%%%%%%%%%%%%%%%%%%%%%%%%%%%%%%%%%%%%%%%%%%%%%%%%%%%%%%%%%%%%%%%%

\begin{section}{Útmutatók}
   \label{Um}
   \Um{Schev1}
	\Um{FolProd1}
\newpage
\end{section}

%%%%%%%%%%%%%%%%%%%%%%%%%%%%%%%%%%%%%%%%%%%%%%%%%%%%%%%%%%%%%%%%%%%%
\hspace{2cm}
%%%%%%%%%%%%%%%%%%%%%%%%%%%%%%%%%%%%%%%%%%%%%%%%%%%%%%%%%%%%%%%%%%%%

\begin{section}{Megoldások}
   \label{Mo}
   \Mo{Schev1}
	\Mo{FolProd1}
   \newpage
\end{section}
%%%%%%%%%%%%%%%%%%%%%%%%%%%%%%%%%%%%%%%%%%%%%%%%%%%
\hspace{2cm}
%%%%%%%%%%%%%%%%%%%%%%%%%%%%%%%%%%%%%%%%%%%%%%%%%%%


\end{spacing}
\end{document}
